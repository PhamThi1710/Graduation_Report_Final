
\section{Cải thiện chất lượng AI}
\begin{itemize}
	\item Tối ưu prompt để sinh bài tập chính xác hơn, tránh tình trạng chung chung hoặc sai ngữ cảnh.
	\item Fine-tune mô hình AI với dữ liệu từ người dùng thực tế (bài làm, phản hồi, feedback) để tăng độ chính xác và cá nhân hóa hơn.
%	\item Phát triển thêm tính năng kiểm tra chính tả, lỗi logic phổ biến hoặc gợi ý code thông minh dựa vào nội dung bài làm.
\end{itemize}

\section{Mở rộng loại bài tập và nội dung học}
\begin{itemize}
	\item Thêm các dạng bài tập mới: trắc nghiệm, điền khuyết, câu hỏi lý thuyết, project-based, \dots
	%\item Hỗ trợ nhiều ngôn ngữ lập trình hơn như Java, Python, JavaScript thay vì chỉ tập trung 1 ngôn ngữ ban đầu.
	\item Tích hợp thêm các nguồn tài liệu học (slide, ebook, video) và tự động phân tích sinh lộ trình học từ đó.
\end{itemize}

\section{Hỗ trợ đánh giá toàn diện và tương tác AI nâng cao}
\begin{itemize}
	\item Cho phép hệ thống đánh giá tiến độ học tập theo nhiều khía cạnh (kiến thức, kỹ năng giải quyết vấn đề, tư duy).
	\item Phát triển tính năng trợ giảng AI giao tiếp theo phong cách hội thoại tự nhiên, không chỉ trả lời theo prompt đơn lẻ.
\end{itemize}

\section{Tối ưu giao diện và trải nghiệm người dùng}
\begin{itemize}
	\item Nâng cao trải nghiệm giao diện: hỗ trợ dark mode, responsive tốt hơn trên mobile.
	\item Thêm các tính năng tương tác như thông báo, nhắc nhở, hoặc gamification để tăng động lực học tập.
    \item Tinh chỉnh các chức năng theo phản hồi từ người dùng thử nghiệm để cải thiện tính thân thiện và dễ sử dụng.
\end{itemize}
