
Trong suốt quá trình phát triển, nhóm dự án đã thực hiện nhiều nhiệm vụ quan trọng để đảm bảo hệ thống học tập trực tuyến được triển khai thành công, đáp ứng yêu cầu và tối ưu hóa trải nghiệm người dùng. Các nhiệm vụ được phân chia theo tuần và người đảm nhiệm, từ việc phát triển các tính năng cơ bản đến việc tích hợp các công nghệ tiên tiến như mô hình ngôn ngữ lớn (LLM) và triển khai hệ thống.

\section{Giai đoạn 1: Phát triển API và Giao diện cơ bản (Tuần 1-3)}

\begin{itemize}[label=-]
    \item Trong tuần đầu tiên, nhóm đã bắt tay vào phát triển các tính năng cốt lõi của hệ thống. 
    \item \textbf{\textit{Phạm Thi}} đảm nhận nhiệm vụ phát triển API cho chức năng đăng nhập và đăng ký, đảm bảo tính bảo mật cho hệ thống thông qua mã hóa mật khẩu và xác thực người dùng. Giao diện UI cho trang đăng nhập và đăng ký cũng được thiết kế đơn giản và dễ sử dụng. Các trường hợp đặc biệt như mật khẩu sai hoặc tài khoản không tồn tại được kiểm tra kỹ lưỡng.
    \item \textbf{\textit{Tuấn Anh}} phụ trách thiết kế giao diện trang giới thiệu của hệ thống, cung cấp thông tin về các tính năng và cách thức hoạt động của hệ thống. Trang landing page này được kiểm tra kỹ lưỡng về tính tương thích và khả năng tương tác trên các nền tảng khác nhau.
    \item \textbf{\textit{Tuấn Anh}} đảm nhiệm việc phát triển API quản lý khóa học cho giảng viên, bao gồm các tính năng thêm, sửa và xóa khóa học, cũng như khả năng upload tài liệu cho từng bài học. Các trường hợp lỗi như tài liệu không đúng định dạng hoặc thiếu thông tin được xử lý rõ ràng.
\end{itemize}

\section{Giai đoạn 2: Quản lý Tiến độ Học tập và Tích hợp API (Tuần 4-6)}

\begin{itemize}[label=-]
    \item Trong tuần tiếp theo, nhóm tiếp tục phát triển các tính năng phục vụ giảng viên và sinh viên. 
    \item \textbf{\textit{Tuấn Anh}} đã hoàn thiện API và giao diện người dùng để giảng viên có thể theo dõi tiến độ học tập của sinh viên. Các tính năng lọc và phân loại học sinh theo tiến độ học tập cũng được thêm vào.
    \item \textbf{\textit{Phạm Thi}} phát triển API để lưu trữ và xử lý dữ liệu học tập của sinh viên, đảm bảo tính chính xác trong việc theo dõi tiến độ học tập. Các API này còn được tích hợp với các hệ thống khác để đồng bộ và cập nhật dữ liệu kịp thời.
    \item \textbf{\textit{Phạm Thi}} phát triển API cho quản lý người dùng (admin), cho phép quản trị viên có thể xem, thêm, sửa và xóa tài khoản của sinh viên và giảng viên. Đồng thời, các quyền truy cập và bảo mật thông tin người dùng cũng được thiết lập và kiểm tra.
\end{itemize}

\section{Giai đoạn 3: Tích hợp Mô hình Ngôn ngữ Lớn (LLM) và Tự động hóa (Tuần 7-10)}

\begin{itemize}[label=-]
    \item Nhóm tiếp tục triển khai các tính năng thông minh, đặc biệt là tích hợp mô hình ngôn ngữ lớn (LLM) để tạo các câu hỏi trắc nghiệm tự động và tài liệu ôn tập.
    \item \textbf{\textit{Tuấn Anh}} phụ trách phát triển API cho việc tạo câu hỏi trắc nghiệm và tài liệu học tập tự động từ nội dung bài giảng, đồng thời kiểm tra độ chính xác của các câu hỏi được tạo ra.
    \item \textbf{\textit{Phương Nam}} phát triển các module hỗ trợ lập trình, bao gồm gợi ý sửa lỗi (Code Hint Module), trợ giúp gỡ lỗi (Debugging Assistant), và giải thích mã nguồn (Explain Code Module). Những tính năng này giúp sinh viên nhận được hỗ trợ trực tiếp trong quá trình học lập trình, tối ưu hóa khả năng học tập.
    \item \textbf{\textit{Tuấn Anh}} phát triển API chấm điểm tự động cho câu trả lời của sinh viên, sử dụng LLM để phân tích và đánh giá các câu trả lời, đảm bảo tính chính xác và khách quan trong việc chấm điểm.
\end{itemize}

