\section{Nhắc lại mục tiêu đề tài}
Mục tiêu của giai đoạn 2 là triển khai hoàn chỉnh các tính năng đã được lên kế hoạch và thiết kế trong giai đoạn 1. Ở giai đoạn 1, mục tiêu của nhóm như sau:
\begin{itemize}
    \item \textbf{Xây dựng hệ thống học tập cá nhân hóa: }Hệ thống sẽ tập trung vào việc cá nhân hóa trải nghiệm học cho từng học viên, phù hợp với khả năng, trình độ, và nhu cầu học tập riêng biệt. Mỗi học viên có một lộ trình học và cách giải thích riêng, giúp họ tiến bộ một cách hiệu quả nhất. Với khả năng phân tích và hiểu ngữ cảnh của LLM, hệ thống có thể đưa ra các phản hồi phù hợp với nhu cầu học tập của từng học viên.
    \item \textbf{Xác định tính khả thi của việc tích hợp LLM trong giáo dục: }Đánh giá khả năng tích hợp mô hình ngôn ngữ lớn vào hệ thống giáo dục, đặc biệt là trong bối cảnh giáo dục lập trình. Chúng tôi sẽ nghiên cứu tính hiệu quả của LLM trong việc cá nhân hóa và nâng cao khả năng tiếp thu kiến thức của học viên.
    \item \textbf{Giải quyết vấn đề lạm dụng LLM trong giải bài tập: }Hiện nay, sinh viên có thể lợi dụng LLM để giải bài tập lập trình mà không thực sự học. Mục tiêu của đề tài là nghiên cứu xem liệu có thể khiến LLM không trực tiếp đưa ra lời giải cho học sinh mà thay vào đó là cung cấp các câu hỏi gợi mở hoặc hướng dẫn giúp học viên tự giải quyết vấn đề. Điều này nhằm phát triển tư duy giải quyết vấn đề của học viên, thay vì chỉ đưa ra câu trả lời.
\end{itemize}
\section{Tóm tắt nội dung đã thực hiện ở Giai đoạn 1}
Giai đoạn 1 của dự án đã hoàn thành việc phát triển các tính năng cơ bản của hệ thống hỗ trợ học tập trực tuyến, tạo nền tảng vững chắc cho các bước triển khai tiếp theo. Các tính năng chính như trang \texttt{dashboard}, \texttt{course list}, \texttt{course detail}, và hệ thống đề xuất bài học đã được triển khai đầy đủ, giúp sinh viên dễ dàng truy cập các khóa học, theo dõi tiến độ học tập và nhận được các gợi ý học tập cá nhân hóa.

Các mô-đun học tập như quiz, bài tập lập trình (code exercises), và tài liệu đọc (reading material) được phát triển để không chỉ cung cấp lý thuyết mà còn tạo cơ hội cho sinh viên thực hành và tự kiểm tra khả năng của mình. Bên cạnh đó, hệ thống cũng ghi nhận các thống kê về thời gian học và tiến độ học tập, từ đó cung cấp các báo cáo chi tiết và đánh giá hiệu quả học tập của sinh viên.

Mặc dù các tính năng cơ bản đã được triển khai, vẫn còn nhiều vấn đề cần giải quyết trong các giai đoạn tiếp theo, bao gồm việc hoàn thiện phần \texttt{authentication}, tối ưu hóa công nghệ và quy trình làm việc, cải thiện độ chính xác của các mô hình ngôn ngữ lớn (LLMs), và mở rộng các tính năng học tập cho giảng viên và quản trị viên. Những vấn đề này sẽ là trọng tâm của giai đoạn 2, với mục tiêu hoàn thiện hệ thống và tối ưu hóa trải nghiệm học tập của người dùng.
