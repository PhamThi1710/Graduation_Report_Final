\section{Tính đúng đắn và hiệu quả của AI (LLM)}
\textbf{Đánh giá đầu ra AI:}
\begin{itemize}
    \item 
    \item 
    \item 
\end{itemize}

\textbf{Tồn tại:}
\begin{itemize}
    \item 
    \item 
\end{itemize}

\textbf{Đánh giá tổng thể:}
\begin{itemize}
    \item \checkmark 
\end{itemize}


\section{Tính đầy đủ và ổn định của chức năng}
\section{Trải nghiệm người dùng (UX/UI)}
\begin{itemize}
    \item Giao diện rõ ràng, dễ sử dụng, đặc biệt cho đối tượng sinh viên CNTT.
    \item Có phản hồi từ người dùng thử nghiệm: tích cực, dễ nắm bắt chức năng.
\end{itemize}

\textbf{Hạn chế:}
\begin{itemize}
    \item Một vài giao diện chưa responsive hoặc cần làm mượt hơn thao tác.
\end{itemize}

\section{Tính mở rộng và bảo trì}
\begin{itemize}
    \item Cấu trúc hệ thống modular, dễ bảo trì và mở rộng thêm bài tập, loại câu hỏi, hoặc mô-đun học mới.
    \item Có thể mở rộng tích hợp thêm loại AI khác như sinh trắc nghiệm, kiểm tra chính tả/lỗi logic.
\end{itemize}

\section{Tính ứng dụng thực tế}
\begin{itemize}
    \item Hệ thống có tiềm năng ứng dụng trong môi trường học tập đại học hoặc nền tảng học trực tuyến cho người học lập trình.
    \item Phù hợp với mô hình giáo dục cá nhân hóa đang phát triển.
\end{itemize}

